%-----------------------------------------------------------------------------------------------------------------------------------------------%
%	ORIGINAL DISCLAIMER
%
%	The MIT License (MIT)
%
%	Copyright (c) 2016 Jan Küster
%
%	Permission is hereby granted, free of charge, to any person obtaining a copy
%	of this software and associated documentation files (the "Software"), to deal
%	in the Software without restriction, including without limitation the rights
%	to use, copy, modify, merge, publish, distribute, sublicense, and/or sell
%	copies of the Software, and to permit persons to whom the Software is
%	furnished to do so, subject to the following conditions:
%	
%	THE SOFTWARE IS PROVIDED "AS IS", WITHOUT WARRANTY OF ANY KIND, EXPRESS OR
%	IMPLIED, INCLUDING BUT NOT LIMITED TO THE WARRANTIES OF MERCHANTABILITY,
%	FITNESS FOR A PARTICULAR PURPOSE AND NONINFRINGEMENT. IN NO EVENT SHALL THE
%	AUTHORS OR COPYRIGHT HOLDERS BE LIABLE FOR ANY CLAIM, DAMAGES OR OTHER
%	LIABILITY, WHETHER IN AN ACTION OF CONTRACT, TORT OR OTHERWISE, ARISING FROM,
%	OUT OF OR IN CONNECTION WITH THE SOFTWARE OR THE USE OR OTHER DEALINGS IN
%	THE SOFTWARE.
%
%	*************	RESOURCES USED	 ********************
%
%	http://tex.stackexchange.com/questions/5718/package-for-pie-charts
%	http://tex.stackexchange.com/questions/183087/draw-colored-world-us-map-in-latex#183138
%	http://www.texample.net/tikz/examples/simple-flow-chart/
%	http://vizualize.me/#
%	http://devnet.kentico.com/getattachment/fd511a92-e164-40f5-ba51-07c228a49fed/Kentico_fortune500_infographics_FINAL.png
%
%-----------------------------------------------------------------------------------------------------------------------------------------------%
%============================================================================%
%
%	TEMPLATE EDITED AND CUSTOMISED BY ANDRÉ BRYNTESSON
%
%============================================================================%

%============================================================================%
%
%	DOCUMENT DEFINITION
%
%============================================================================%

%we use article class because we want to fully customize the page
\documentclass[10pt,A4]{article}	


%----------------------------------------------------------------------------------------
%	ENCODING
%----------------------------------------------------------------------------------------

%we use utf8 since we want to build from any machine
\usepackage[utf8]{inputenc}		

%----------------------------------------------------------------------------------------
%	LOGIC
%----------------------------------------------------------------------------------------

\usepackage{xifthen}
\usepackage{calc}

%----------------------------------------------------------------------------------------
%	FONT
%----------------------------------------------------------------------------------------

% some tex-live fonts - choose your own

%\usepackage[defaultsans]{droidsans}
%\usepackage[default]{comfortaa}
%\usepackage{cmbright}
\usepackage[default]{raleway}
%\usepackage{fetamont}
%\usepackage[default]{gillius}
%\usepackage[light,math]{iwona}
%\usepackage[thin]{roboto} 
\usepackage{hyperref}
\hypersetup{
    colorlinks=true,
    linkcolor=blue,
    filecolor=magenta,      
    urlcolor=cyan,
}

% set font default
\renewcommand*\familydefault{\sfdefault} 	
\usepackage[T1]{fontenc}

% more font size definitions
\usepackage{moresize}		

% awesome font
\usepackage{fontawesome5}


%----------------------------------------------------------------------------------------
%	PAGE LAYOUT  DEFINITIONS
%----------------------------------------------------------------------------------------

%debug page outer frames
%\usepackage{showframe}			

%define page styles using geometry
\usepackage[a4paper]{geometry}		

% for example, change the margins to 2 inches all round
\geometry{top=1.5cm, bottom=1cm, left=1cm, right=1cm} 	

% use customized header
\usepackage{fancyhdr}				
\pagestyle{fancy}

%less space between header and content
\setlength{\headheight}{-5pt}		

% customize header entries
\lhead{}
\rhead{}
\chead{}

%indentation is zero
\setlength{\parindent}{0mm}

%----------------------------------------------------------------------------------------
%	TABLE /ARRAY DEFINITIONS
%---------------------------------------------------------------------------------------- 

%extended aligning of tabular cells
\usepackage{array}

% custom column width
\newcolumntype{x}[1]{%
>{\raggedleft\hspace{0pt}}p{#1}}%


%----------------------------------------------------------------------------------------
%	GRAPHICS DEFINITIONS
%---------------------------------------------------------------------------------------- 

\usepackage{graphicx}
\usepackage{wrapfig}

% for drawing graphics and charts
\usepackage{tikz}
\usetikzlibrary{shapes, backgrounds}

% use to vertically center content
% credits to: http://tex.stackexchange.com/questions/7219/how-to-vertically-center-two-images-next-to-each-other
\newcommand{\vcenteredinclude}[1]{\begingroup
\setbox0=\hbox{\includegraphics{#1}}%
\parbox{\wd0}{\box0}\endgroup}

% use to vertically center content
% credits to: http://tex.stackexchange.com/questions/7219/how-to-vertically-center-two-images-next-to-each-other
\newcommand*{\vcenteredhbox}[1]{\begingroup
\setbox0=\hbox{#1}\parbox{\wd0}{\box0}\endgroup}

%----------------------------------------------------------------------------------------
%	ICON-SET EMBEDDING
%---------------------------------------------------------------------------------------- 

% at this point we simplify our icon-embedding by simply referring to a set of png images.
% if you find a good way of including svg without conflicting with other packages you can
% replace this part

\newcommand{\icon}[2]{\colorbox{thirdcol}{\makebox(#2, #2){\textcolor{titletext}{\csname fa#1\endcsname}}}}	%icon shortcut
\newcommand{\icontext}[3]{ 						%icon with text shortcut
	\vcenteredhbox{\icon{#1}{#2}} \vcenteredhbox{\textcolor{textcol}{#3}}
}

\newcommand{\icox}[4]{\colorbox{#3}{\makebox(#2, #2){\textcolor{#4}{\csname fa#1\endcsname}}}}	%icon shortcut
\newcommand{\iconbox}[5]{ 						%icon with text shortcut
	\vcenteredhbox{\icox{#1}{#2}{#3}{#5}} \vcenteredhbox{\textcolor{#5}{\colorbox{#3}{#4}}}
}

\newcommand{\icoxfa}[4]{\colorbox{#3}{\makebox(#2, #2){\textcolor{#4}{\faIcon{#1}}}}}	%icon shortcut
\newcommand{\iconboxfa}[5]{ 						%icon with text shortcut
	\vcenteredhbox{\icoxfa{#1}{#2}{#3}{#5}} \vcenteredhbox{\textcolor{#5}{\colorbox{#3}{#4}}}
}

\newcommand{\iconref}[2]{\textcolor{#1}{\csname fa#2\endcsname}}	%icon shortcut
\newcommand{\iconweb}[5]{ 						%icon with text ref shortcut
	\href{https://www.#5}{\vcenteredhbox{\iconref{#1}{#2}} \textcolor{#3}{#4}}
}

%----------------------------------------------------------------------------------------
%	Color DEFINITIONS
%---------------------------------------------------------------------------------------- 

\usepackage{xcolor}

%defineColors
\definecolor{pyblue}{HTML}{4B8BBE}
% \definecolor{pyblue}{HTML}{306998}

\definecolor{lpyellow}{HTML}{FFE873}
\definecolor{pyellow}{HTML}{FFD43B}

\definecolor{pygray}{HTML}{646464}

\definecolor{sorange}{RGB}{255,150,0}
\definecolor{lblue}{RGB}{0,178,255}
\definecolor{darkblue}{RGB}{0,80,130}
\definecolor{darkerblue}{RGB}{0,100,160}
\definecolor{lgray}{RGB}{0,120,200}
\definecolor{powderblue}{RGB}{190,220,255}
\definecolor{darkestblue}{RGB}{0,50,80}
\definecolor{petrol}{RGB}{0,140,140}
\definecolor{black}{RGB}{0,0,0}


%main color
\colorlet{maincol}{sorange}

%secondary colors
%\colorlet{secondcol}{lblue}
\colorlet{secondcol}{pygray}
\colorlet{thirdcol}{darkblue}
\colorlet{fourthcol}{pyblue}
\colorlet{fifthcol}{lgray}
\colorlet{sixthcol}{darkestblue}
\colorlet{seventhcol}{black}
\colorlet{eighthcol}{petrol}

%background color
\colorlet{bgcol}{powderblue}

%textcolor
\colorlet{textcol}{darkestblue}

%titletextcolor
\colorlet{titletext}{white}

%sectioncolor
\colorlet{sectcol}{white}

%set a background col for whole page
\pagecolor{bgcol}


%----------------------------------------------------------------------------------------
% 	HEADER
%----------------------------------------------------------------------------------------

% remove top header line
\renewcommand{\headrulewidth}{0pt} 

%remove botttom header line
\renewcommand{\footrulewidth}{0pt}	  	

%remove pagenum
\renewcommand{\thepage}{}	

%remove section num		
\renewcommand{\thesection}{}			


%----------------------------------------------------------------------------------------
%
% 	TIKZ GRAPHICS
%
%----------------------------------------------------------------------------------------


% the chart graphics are outsourced into own files

%----------------------------------------------------------------------------------------
% 	PIE CHART
%----------------------------------------------------------------------------------------
\input{./g/piechart.tex}

%----------------------------------------------------------------------------------------
% 	BAR CHART
%----------------------------------------------------------------------------------------
\input{./g/barchart.tex}


%----------------------------------------------------------------------------------------
% 	BUBBLE CHART
%----------------------------------------------------------------------------------------
\input{./g/bubbles.tex}

%----------------------------------------------------------------------------------------
% 	SQUARE CHART
%----------------------------------------------------------------------------------------
\input{./g/squares.tex}

%----------------------------------------------------------------------------------------
% 	TIMELINE VERTICAL & HORIZONTAL CHART
%----------------------------------------------------------------------------------------
\input{./g/timeline.tex}

%----------------------------------------------------------------------------------------
% 	FACT BUBBLE
%----------------------------------------------------------------------------------------
\input{./g/factbubble.tex}


%----------------------------------------------------------------------------------------
%	custom sections
%----------------------------------------------------------------------------------------

% create a coloured box with arrow and title as cv section headline
% param 1: section title
%
\newcommand{\cvsection}[2] {
\textcolor{titletext}{\uppercase{\textbf{#1}}}
}

\newcommand{\cvsect}[4]{
	\textcolor{#3}{\hrule}
	\colorbox{#3}{ {\cvsection{#1}{#4}}}
}

% create a coloured arrow with title as cv meta section section
% param 1: meta section title
%
\newcommand{\metasection}[2] {
	\begin{tabular*}{1\textwidth}{ l l }
		#1&#2\\[12pt]
	\end{tabular*}
}

%----------------------------------------------------------------------------------------
%	 CV EVENT
%----------------------------------------------------------------------------------------

% creates a stretched box as 
\newcommand{\cveventmeta}[2] {
	\mbox{\mystrut \hspace{87pt}\textit{#1}}\\
	#2
}

%----------------------------------------------------------------------------------------
% STRUTS AND RULES
%----------------------------------------- -----------------------------------------------

% custom strut
\newcommand{\mystrut}{\rule[-.3\baselineskip]{0pt}{\baselineskip}}

% colored rule and text for chart legends, wrapped in parbox
% param 1: text
% param 2: width in cm or pt, em ...
% param 3: color
\newcommand{\legend}[3]{\parbox[t]{#2}{\textcolor{#3}{\rule{#2}{4pt}}\\#1}}

%----------------------------------------------------------------------------------------
% CUSTOM LOREM IPSUM
%----------------------------------------------------------------------------------------
\newcommand{\lorem}{Lorem ipsum dolor sit amet, consectetur adipiscing elit. Donec a diam lectus.}


%============================================================================%
%
%
%
%	DOCUMENT CONTENT
%
%
%
%============================================================================%
\begin{document}


%use our custom fancy header definitions
\pagestyle{fancy}	


%----------------------------------------------------------------------------------------
%	TITLE HEADLINE
%----------------------------------------------------------------------------------------
\mystrut
\vspace{-12pt}

\hspace{-16pt}\begin{tabular*}{1\textwidth}{ c c c}
	\parbox[c]{0.59\linewidth}{
		\colorbox{thirdcol}{\huge{\textcolor{titletext}{\textbf{\uppercase{André Bryntesson}}}}}\\
	}&
	\parbox{0.40\textwidth}{
	\icontext{BirthdayCake}{12}{19900312-5615}\\
	\icontext{MapMarker}{12}{Uppsala}\\
	\href{callto:+212622902221}{\icontext{Mobile}{12}{+46 73 214 75 35}}\\
	\href{mailto:bryntesson.andre@gmail.com}{\icontext{At}{12}{bryntesson.andre@gmail.com}}\\
	}
\end{tabular*}

% manage space by reducing font size
\small

\vspace{5pt}

\hspace{-10pt}\begin{minipage}{0.59\textwidth}

%----------------------------------------------------------------------------------------
%	FACTS
%----------------------------------------------------------------------------------------

	\mbox{
		\parbox[c][3cm][c]{0.29\textwidth}{
			\textcolor{textcol}{Jag är statsvetare och ämneslärare med många års erfarenhet av forskning inom utbildningssociologi.}
				}
		\hspace{10pt}
		\parbox[c][3cm][c]{0.32\textwidth}{
			\begin{center}
			\factbubble{\small{\textcolor{titletext}{\textbf{Statsvetare}}}\\\small{\textcolor{titletext}{\textbf{Ämneslärare}}}}{1}{maincol}{titletext}{thirdcol}\\
			\textcolor{textcol}{\textbf{Huvudförfattare till forskningsrapporter på uppdrag av UKÄ, UHR och DELMI.
}}
			\end{center}
		}
		\hspace{10pt}
		\parbox[c][3cm][c]{0.29\textwidth}{
			\textcolor{textcol}{Söker karriärväg utanför akademin, med möjlighet till fortsatt god balans mellan arbete och fritid.}
		}
	}
	\vspace{50pt}

%----------------------------------------------------------------------------------------
%	SKILLS
%----------------------------------------------------------------------------------------

	\cvsect{Kunskaper och färdigheter}{0.49}{thirdcol}{textcol}\\[10pt]
	\mbox{
		
		% TEXT BOX
		\parbox[b][140pt][c]{0.38\textwidth}{
			\textcolor{textcol}{Jag är generalist och har goda kunskaper om internationalisering och breddad rekrytering inom högskolan. Genom studier och arbete har jag fått erfarenhet av olika typer av analysmetoder och data. Jag har presenterat mitt arbete på konferenser såväl i Sverige som i Luxemburg, Nederländerna, Estland och Tyskland.}
			
			% LANGUAGES
			\cvsect{Språk}{0.49}{thirdcol}{textcol}\\[4pt]
			\iconbox{Language}{12}{thirdcol}{Svenska (modersmål)}{titletext}\\
			\iconbox{Language}{12}{thirdcol}{Engelska (akademisk)}{titletext}\\
		}

		% BAR CHART
	    \begin{barchart}{5}{4}{sectcol}{textcol}{sectcol}{maincol}{secondcol}{thirdcol}
	    	\baritem{80}{Skrivande}{70}{0}{0}
	    	\baritem{80}{Klarspråk}{0}{0}{45}
	    	\baritem{80}{Presentation}{80}{0}{0}
	    	\baritem{80}{Studiedesign}{0}{0}{65}
	    	\baritem{80}{Kvalitativ analys}{0}{0}{60}
	    	\baritem{80}{Dataanalys}{0}{0}{45}
	    	\baritem{80}{Programmering}{0}{0}{10}
	        \baritem{80}{Konstruktiv kritik}{80}{0}{0}
	    	\baritem{80}{Självständighet}{70}{0}{0}
	    	\baritem{80}{Tidsplanering}{0}{0}{65}
	    	\baritem{80}{Ekonomi/adm.}{0}{0}{20}
		\end{barchart}
		}
	\begin{center}
	\begin{tikzpicture}
		\draw[draw=sectcol,dashed, opacity=0.0] (4,0) -- (-4,0);
	\end{tikzpicture}
	\end{center}
	\hspace{12pt}\\


	% BAR CHART
	\mbox{\hspace{-12pt}
		\begin{barchart}{10}{4}{sectcol}{textcol}{sectcol}{maincol}{secondcol}{thirdcol}
			\baritem{80}{SPSS}{0}{0}{40}
			\baritem{80}{SPAD}{0}{0}{30}
			\baritem{80}{R}{0}{0}{15}
			\baritem{80}{MS Office}{0}{0}{55}
			\baritem{80}{Zotero}{0}{0}{35}
		\end{barchart}
		\hspace{12pt}

		% TEXTBOX
		\parbox[b][80pt][c]{0.58\textwidth}{\textcolor{textcol}{Jag har erfarenhet av att sammanställa och analysera  policy och forskning. Utifrån mikrodata och enkätdata har jag även genomfört regressionsanalyser i SPSS och geometriska dataanalyser i SPAD. Viss erfarenhet har jag också av textanalys och intervjuer. Just nu lär jag mig dataanalys i R.}}
	}
	\vspace{10pt}

%----------------------------------------------------------------------------------------
%	ACTIVITIES
%----------------------------------------------------------------------------------------
	\cvsect{Fritid}{0.49}{thirdcol}{textcol}\\[5pt]
	\mbox{
		
		% FACT BUBBLE
		\parbox[b][3cm][c]{2.5cm}{
			\factbubble{\HUGE{\textcolor{sectcol}{\textbf{8}}}\\\large{\textcolor{sectcol}{\textbf{år}}}}{0.85}{maincol}{sectcol}{thirdcol}\\
			\textcolor{textcol}{Så länge drev jag kollektiv med internationella studenter.}\\
		}

		% TEXTBOX
		\parbox[b][80pt][c]{0.04\textwidth}{\textcolor{textcol}{}}
		
		% PIE CHART	
		\begin{piechart}{360}{1.6}{bgcol}{textcol}{sectcol}
			\slice{20}{Matlagning}{thirdcol}
			\slice{10}{Personlig utveckling}{maincol}
			\slice{20}{Klättring}{fifthcol}
			\slice{10}{Friluftsliv}{sixthcol}
			\slice{10}{Läsning}{fourthcol}
			\slice{15}{Brädspel}{secondcol}
			\slice{10}{Foto}{seventhcol}
			\slice{5}{TV-serier}{eighthcol}
		\end{piechart}\\
	}
\end{minipage}
\begin{minipage}{0.05\textwidth}
	\begin{center}
		\begin{tikzpicture}
			\draw[draw=sectcol,dashed, opacity=0.5] (0,-12) -- (0,12);
		\end{tikzpicture}
	\end{center}
\end{minipage}
\begin{minipage}{0.4\textwidth}
%----------------------------------------------------------------------------------------
%	EDUCATION / PROJECT / EXPERIENCE
%----------------------------------------------------------------------------------------
\cvsect{Utbildning och erfarenhet}{0.49}{thirdcol}{textcol}\\ [16pt]

\hspace{55pt}\mbox{\legend{Utbildning}{1.5cm}{maincol} \legend{Arbete}{1.1cm}{secondcol} \legend{Publikation}{1.7cm}{thirdcol} \legend{Event}{0.9cm}{fifthcol}}

\vspace{-50pt}
\begin{center}

% TIMELINE
\begin{timelinevertical}{2010}{2022}{20}{\linewidth}{0.1}

\cvevent{9/2010}{6/2013}{Kandidatstudier, statskunskap \newline \textcolor{pygray}{Uppsala universitet \newline Biämnen: Sociologi, NEK, Retorik}}{0.5}{0}{maincol}{0.4}

\cvevent{5/2013}{5/2013}{Kandidatuppsats, statskunskap\newline \textcolor{pygray}{Nominerad till Juseks nationella tävling som institutionens bästa uppsats 2013}}{0.5}{0}{fifthcol}{0.4}

\cvevent{11/2012}{10/2013}{Forskningsassistent \newline \textcolor{pygray}{Statsvetenskapliga institutionen, UU \newline Datainsamling till konfliktdatabas}}{0.5}{-0.5}{secondcol}{0.2}

\cvevent{9/2013}{1/2015}{Korta ämneslärarprogrammet \newline \textcolor{pygray}{Uppsala universitet \newline Inriktning samhällskunskap i gymnasiet}}{0.5}{0.9}{maincol}{0.4}

\cvevent{03/2015}{03/2016}{Sociologi, fristående kurser \newline \textcolor{pygray}{Uppsala universitet \newline För lärarlegitimation i sociologi}}{0.5}{-0.25}{maincol}{0.4}

\cvevent{9/2015}{12/2021}{Forskningsassistent \newline \textcolor{pygray}{EDU, Utbildningssociologi, UU \newline Assistent till prof. Mikael Börjesson}}{0.5}{0.3}{secondcol}{0.25}

\cvevent{1/2017}{6/2017}{Lärare i samhällskunskap \newline \textcolor{pygray}{Grillska Gymnasiet, Uppsala \newline Deltidsarbete, avböjde förlängning}}{0.5}{-0.5}{secondcol}{0.1}

\cvevent{9/2017}{9/2017}{Föreläsare på masternivå \newline \textcolor{pygray}{Transnationella utbildningsstrategier}}{0.5}{-0.4}{fifthcol}{0.1}

\cvevent{1/2018}{1/2018}{\href{https://www.edu.uu.se/digitalAssets/687/c_687498-l_1-k_bryntesson--a.-det-svenska-studieavgiftssystemets-ekonomiska-logik---sihe-rapport-1.pdf}{SIHE-rapport, Nr 1} \newline \textcolor{pygray}{Om studieavgiftsreformen}}{0.5}{0}{thirdcol}{0.4}

\cvevent{10/2018}{10/2018}{\href{https://www.uhr.se/globalassets/_uhr.se/publikationer/2018/rapport-fran-sverige-med-erasmus-en.pdf}{UHR-rapport 2018:13} \newline \textcolor{pygray}{Om utresande Erasmus-studenter}}{0.5}{-0.2}{thirdcol}{0.4}

\cvevent{3/2019}{3/2019}{Presentatör, plenum, Tallinn \newline \textcolor{pygray}{Nordic-Baltic Migration Conference}}{0.5}{0.1}{fifthcol}{0.1}

\cvevent{5/2019}{5/2019}{\href{https://www.delmi.se/media/1xtf3tei/delmi-rapport-2019_4.pdf}{DELMI-rapport 2019:4} \newline \textcolor{pygray}{Om studieavgifter och studentflöden}}{0.5}{0.8}{thirdcol}{0.4}

\cvevent{12/2019}{12/2019}{\href{https://www.uhr.se/globalassets/_uhr.se/publikationer/2019/uhr-svenska-studenter-i-erasmus-plus.pdf}{UHR-rapport 2019:6} \newline \textcolor{pygray}{Om Erasmus-studenters bakgrund}}{0.5}{0.8}{thirdcol}{0.4}

\cvevent{6/2021}{6/2021}{\href{http://www.skeptron.uu.se/broady/sec/sec-63.pdf}{SEC-rapport, Nr 63, åt UKÄ} \newline \textcolor{pygray}{Om policy för breddad rekrytering}}{0.5}{-0.6}{thirdcol}{0.4}

\cvevent{10/2021}{10/2021}{\href{http://www.skeptron.uu.se/broady/sec/sec-64.pdf}{SEC-rapport, Nr 64, åt UKÄ} \newline \textcolor{pygray}{Om forskning om snedrekrytering}}{0.5}{-0.1}{thirdcol}{0.4}

\end{timelinevertical}
\end{center}
\vspace{-40pt}\end{minipage}
%============================================================================%
%
%
%
%	DOCUMENT END
%
%
%
%============================================================================%
\end{document}
